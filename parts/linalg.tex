% !TeX root = ../RegT4.tex
% vim: ts=2 sw=2 spell:

\section{Linear Algebra}

% \subsection{Matrix Multiplication}

\subsection{Determinant}

The determinant is a measure of how the unit square (\(n = 2\)), cube (\(n = 2\)) or hypercube (\(n > 3\)) is scaled by the linear transformation. The determinant of a \(2 \times 2\) matrix is computed using the ``fish rule'':
\[
	\det
	\begin{bmatrix}
		a & b \\
		c & d
	\end{bmatrix}
	=
	ad - bc.
\]
By hand, for a \(3 \times 3\) matrix, Sarrus' rule is probably the fastest method:
\[
	\det
	\underbrace{
		\begin{bmatrix}
			a & b & c \\
			d & e & f \\
			g & h & i
		\end{bmatrix}
	}_{\mx{A}}
	\leadsto
	\begin{bmatrix}
		\begin{tikzpicture}[
				outer sep = 0,
				inner sep = 2pt,
			]
			\matrix (m) [
				matrix of nodes, row sep = 3pt, column sep = 5pt,
				nodes = {
					font = \itshape,
				}
			] {
				a & b & c & |[gray]| a & |[gray]| b \\
				d & e & f & |[gray]| d & |[gray]|   \\
				g & h & i & |[gray]| g & |[gray]| h \\
			};
			\begin{scope}[
					line width = 7pt, opacity = .2, line cap = round,
				]
				\draw [red] (m-1-1.center) -- (m-3-3.center);
				\draw [red] (m-1-2.center) -- (m-3-4.center);
				\draw [red] (m-1-3.center) -- (m-3-5.center);
				\draw [blue] (m-1-3.center) -- (m-3-1.center);
				\draw [blue] (m-1-4.center) -- (m-3-2.center);
				\draw [blue] (m-1-5.center) -- (m-3-3.center);
			\end{scope}
		\end{tikzpicture}
	\end{bmatrix},
\]
the red lines are added, while the blue ones are subtracted:
\[
	\det \mx{A} = aei + bfg + cdh - ceg - afh - bdi.
\]

\subsection{Inverse Matrix}

The inverse of a \(2 \times 2\) matrix is
\[
	\begin{bmatrix}
		a & b \\
		c & d
	\end{bmatrix}^{-1}
	=
	\frac{1}{ad - bc}
	\begin{bmatrix}
		d & -b \\
		-c & a
	\end{bmatrix},
\]
I remember it as ``transposed on the wrong diagonal and add minuses (and divide by the determinant)''. For bigger matrices, there are the method of Gaussian elimination: 
\[
	\begin{bmatrix} \mx{A} \,|\, \mx{I} \, \end{bmatrix}
		\xrightarrow{\text{Gauss}}
	\begin{bmatrix} \, \mx{I} \,|\, \mx{A}^{-1} \end{bmatrix},
\]
or Cramers' rule \( \mx{A}^{-1} = \adj \mx{A} / \det \mx{A}\).

\subsection{Eigenvalues and Eigenvectors}

The eigenvalue problem is to find the vectors that are unaffected by a matrix \(A\) (up to a scalar factor):
\[
	\mx{A} \vec{v} = \lambda \vec{v} \iff (\mx{A} - \lambda \mx{I}) \vec{v} = \vec{0}.
\]
There are \(n\) numbers \(\lambda\), the \emph{eigenvalues}, and equally many \emph{eigenvectors} \(\vec{v}\). This problem is solved by first finding the \(\lambda\)'s and then later the eigenvectors.

In the right expression, we need a matrix \(\mx{A} - \lambda \mx{I}\) to move any vector to the origin, this is accomplished by forcing
\[
	\det (\mx{A} - \lambda \mx{I}) = 0,
\]
and solving for the \(\lambda\)'s. Then to find the eigenvectors we put back each eigenvalue \(\lambda_i\) into the original equation to obtain a system of equations
\[
	(\mx{A} - \lambda_i \mx{I}) \vec{v}_i = \vec{0}.
\]
This system of equations is solved for the components of \(\vec{v}_i\), and it is overdetermined. This is because there are infinitely many eigenvectors each eigenspace of \(\mx{A}\).

For a \(2 \times 2\) matrix the eigenvalue problem is a parabola, so there is a shortcut:
\begin{align*}
	m &= \frac{1}{2} \tr \mx{A}, & p &= \det \mx{A},
\end{align*}
then \(\lambda_{1,2} = m \pm \sqrt{m^2 - p}\).

\subsection{Diagonalization}

If a \(n \times n\) matrix \(\mx{A}\) has \(n\) distinct eigenvalues, it is \emph{diagonalizable}. To diagonalize \(\mx{A}\), first we compute the eigenvalues \(\lambda_1, \ldots, \lambda_n\) and the eigenvectors \(\vec{v}_1, \ldots, \vec{v}_n\), then a matrix is formed by using the eigenvectors as rows:
\[
	\mx{T} = \begin{bmatrix}
		\vec{v}_1 & \vec{v}_2 & \dots & \vec{v}_n
	\end{bmatrix}.
\]
This matrix is then used for a change of bases into the eigenbasis of \(\mx{A}\), where it is a diagonal matrix:
\[
	\mx{\Lambda} = \mx{T} \mx{A} \mx{T}^{-1} = \begin{bmatrix}
		\lambda_1 \\
		& \ddots \\
		& & \lambda_n
	\end{bmatrix}.
\]

\subsection{Jordan Normal Form}

\subsection{Caley-Hamilton Theorem}

\subsection{Useful Matrices and Products}

\paragraph{Quadratic Form}

The quadratic product is an expression of the form \(\vec{x}^* \mx{Q} \vec{x}\). Let \(\vec{x} \in \mathbb{C}^{2\times 1}\) be a complex column vector and \(a, b, q\) be complex numbers:
\[
	\vec{x}^*
	\underbrace{
		\begin{bmatrix}
			a & q \\ q & b
		\end{bmatrix}
	}_{\mx{Q}}
	\vec{x}
	=
	a |x_1|^2 + q(x_1^* x_2 + x_1 x_2^*) + b |x_2|^2.
\]
If working only with real numbers:
\[
	\vec{x}^\mathsf{T} \mx{Q} \vec{x} = a x_1^2 + 2q x_1 x_2 + b x_2^2.
\]
For bigger (real) \(n\) dimensional vectors:
\[
	\vec{x}^\mathsf{T} \mx{Q} \vec{x} = \sum_{i,j}^{n^2} Q_{ij} \, x_i x_j.
\]

\paragraph{Column Row Product}

For two complex numbers \(a\) and \(b\):
\[
	\vec{x} =
	\begin{bmatrix}
		a \\ b
	\end{bmatrix}
	\quad
	\leadsto
	\quad
	\vec{x} \vec{x}^* =
	\begin{bmatrix}
		|a|^2 & a b^* \\
		a^* b & |b|^2
	\end{bmatrix}.
\]

\paragraph{Rotation Matrices}

In 2 dimensions:
\[
	\mx{R}(\theta) =
	\begin{bmatrix}
		\cos \theta & -\sin \theta \\
		\sin \theta & \cos \theta
	\end{bmatrix}
	%,
	.
\]
% while in 3 dimensions:
% \begin{align*}
% 	\mx{R}_x(\theta) &=
% 	\begin{bmatrix}
% 		1 & 0 & 0 \\
% 		0 & \cos \theta & -\sin \theta \\
% 		0 & \sin \theta & \cos \theta  \\
% 	\end{bmatrix}, \\
% 	\mx{R}_y(\theta) &=
% 	\begin{bmatrix}
% 		\cos \theta & 0 & -\sin \theta \\
% 		0 & 1 & 0 \\
% 		\sin \theta & 0 & \cos \theta \\
% 	\end{bmatrix}, \\
% 	\mx{R}_z(\theta) &=
% 	\begin{bmatrix}
% 		\cos \theta & -\sin \theta & 0 \\
% 		\sin \theta & \cos \theta & 0 \\
% 		0 & 0 & 1
% 	\end{bmatrix}.
% \end{align*}
Rotation matrices belong to the special orthogonal group, therefore
\[
	\mx{R}\mx{R}^\mathsf{T} = \mx{I}
	\iff
	\mx{R}^{-1} = \mx{R}^\mathsf{T}
	\text{ and }
	\det \mx{R} = 1.
\]
