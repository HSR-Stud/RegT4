% !TeX root = ../RegT4.tex
% vim: ts=2 sw=2 spell:

\section{Discrete-Time Modelling}

\begin{figure}
	\centering
	% vim: ts=2 sw=2 spell:
\begin{tikzpicture}
	\draw[gray, thick, dashed] (3.5, 0) -- (-3.5, 0);

	\draw[gray, thick, dashed] (0, 2.5) -- 
		node[above, midway, sloped, rotate = 180, fill = white, text = black] {Continuous}
		node[below, midway, sloped, rotate = 180, fill = white, text = black] {Discrete} (0, -2.5);

	\matrix (m) at (0, 0) [
		matrix of nodes, row sep = 2cm, column sep = 3.25cm,
		nodes = {
			draw, thick, outer sep = 1mm, fill = white,
		},
	] {
		\(G_p(s)\) & \(\hat{G}_p(z)\) \\
		\(G_c(s)\) &
			\node[xshift = -9mm]  (Gci) {\(G_{c,i}(z)\)};
			\node[yshift = 9mm] (Gcd) {\(G_{c,d}(z)\)}; \\
	};

	\node[left = 10mm, rotate = 90, anchor = center] at (m-1-1) {Plant};
	\node[left = 10mm, rotate = 90, anchor = center] at (m-2-1) {Controller};

	\draw[thick, ->] (m-1-1) --
		node[midway, above = 4mm, fill = white]
		{Discretize Plant (ZOH)} (m-1-2);

	\draw[thick, ->] (m-1-2) to[out = -90, in = 90] (Gcd);

	\draw[thick, ->] (m-1-1) -- (m-2-1);

	\draw[thick, ->] (m-2-1) to[out = 0, in = 180] 
		node[pos = .7, below = 4mm, fill = white]
		{Discretize Controller (Tustin)} (Gci);

	\draw[line width = 1mm, red, opacity = .3, ->] (m-1-1) to[out = -90, in = 180]
		node[pos = .5, sloped, text = red!80!black, opacity = 1, above] {Indirect} (Gci);

	\draw[line width = 1mm, blue, opacity = .3, ->] (m-1-1) to[out = 0, in = 90]
		node[pos = .6, sloped, text = blue!80!black, opacity = 1, below] {Direct} (Gcd);
\end{tikzpicture}

	\caption{
		Direct vs indirect design. In general \(G_{c,i}(z) \neq G_{c,d}(z)\).
		\label{fig:direct-vs-indirect}
	}
\end{figure}

\subsection{Sampling}

In discrete time the system works with sampled signals. The sampling time \(T\) must be chosen such that the sampling theorem (Nyquist Shannon) is satisfied, i.e.
\[
	\omega_s = \frac{2\pi}{T} \geq 2 \omega_r,
\]
where \(\omega_r\) is the is the highest frequency component in the reference signal \(r(t)\).

In practice sampling means using an analog to digital converter (ADC), but mathematically the sampling is done through a multiplication with a Dirac comb
\[
	\sum_k \delta(t - kT).
\]

The converse operation, which is done with a digital to analog converter (DAC), is represented in many ways depending on the type of reconstruction. The simplest reconstruction is the zero order hold (ZOH) reconstruction, which keeps the signal fixed at the value of the last digital sample until the new one comes.

\subsubsection{Choosing \(T\) Empirically}

\todo{Stuff from skript p. 230, parameters from Takahashi et al.}

% \begin{figure}
% 	\centering
% 	\skelfig[width = .9\linewidth, height = 3cm]
% 	\caption{
% 		Measurements to choose \(T\) empirically.
% 	}
% \end{figure}

\subsection{Indirect Design}

In the indirect design we create a controller in the continuous time domain and then discretize it (see figure \ref{fig:direct-vs-indirect}). This is okay as long as the sampling theorem is satisfied.

Thus, given a plant with transfer function \(G_p(s)\) we create a continuous time controller \(G_c(s)\). Then to discretize it we can choose how to approximate the integrator element from one of the following 3 options:
\begin{align*}
	s & = \frac{z - 1}{T}                & z & = 1 + sT                && \text{(Euler)},   \\
	s & = \frac{z - 1}{Tz}               & z & = \frac{1}{1 - sT}      && \text{(Forward)}, \\
	s & = \frac{2}{T}\frac{z - 1}{z + 1} & z & = \frac{2 + sT}{2 - sT} && \text{(Tustin)}.
\end{align*}
The last one (Tustin) is the most popular. To obtain the discretized controller transfer function \(\hat{G}_c(z)\) we substitute \(s\) with the chosen approximation (for example Tustin):
\[
	\hat{G}_c(z) = G_c \left(s = \frac{2}{T} \frac{z - 1}{z + 1}\right).
\]

\subsection{ZOH Discretization (Direct Design)}

\subsubsection{Discretization of a Transfer Function}

In the direct design the plant's continuous time model \(G_p(s)\) is discretized. If we assume that the DAC is a ZOH, when the input is a discrete Dirac impulse in continuous time it becomes a rectangle of length \(T\):
\[
	u(t) = \varepsilon(t) - \varepsilon(t - T)
	\leadsto U(s) = \frac{1 - e^{-sT}}{s}.
\]
Then we compute the output of the system to this input in continuous time \(U(s)G_p(s)\), inverse Laplace transform it, and then sample it in the time domain. By taking the \(z\) transform of the result we obtain a discretized transfer function 
\[
	\hat{G}_p(z) = \left( 1 + z^{-1} \right) \ztr \left\{
		\left( \laplace^{-1} \left\{
			\frac{G(s)}{s} \right\}
		\right) \Bigg |_{t = kT}
	\right\}.
\]
ZOH discretizations for a few common type of plants are listed in table \ref{tab:zoh-plants}.

\begin{table}[h]
	\centering
	% \begin{tabularx}{\linewidth}{
	\renewcommand{\arraystretch}{1.9}
	\begin{tabular}{
			>{\(\displaystyle}l<{\)}
			>{\(\displaystyle}l<{\)}
			>{\(\displaystyle}l<{\)}
			>{\(\displaystyle}l<{\)}
		}
		\toprule
		\textbf{Plant } G_p(s) & \textbf{ZOH } \hat{G}_p(z) &
		\textbf{Plant } G_p(s) & \textbf{ZOH } \hat{G}_p(z) \\
		\cmidrule(r){1-2} \cmidrule(l){3-4}
		K                   & \frac{Kz}{z - 1}                       &
		\frac{1}{s}         & \frac{T}{z - 1}                        \\
		\frac{1}{s^2}       & \frac{T}{2} \frac{z + 1}{(z - 1)^2}    &
		\frac{K}{s\tau + 1} & \frac{1 - e^{-T/\tau}}{z - e^{T/\tau}} \\
		\bottomrule
	\end{tabular}
	\caption{
		ZOH discretization of a few plant types.
		\label{tab:zoh-plants}
	}
\end{table}

\subsubsection{Discretization of a State Space Model}

\todo{Stuff from skript p. 250.}

\subsubsection{Root Locus Method}

\todo{Stuff from skript p. 254.}

\subsubsection{Pole Placement Design}

Another way to design a controller is to simply state that the closed \(G_f(z)\) has to have the poles at some determined locations. Then since in a closed feedback loop with a plant \(G_p(z)\) and a controller \(G_c(z)\):
\[
	G_f(z) = \frac{G_c(z) G_p(z)}{1 + G_p(z) G_c(z)},
\]
by solving for \(G_c(z)\):
\[
	G_c(z) = \frac{1}{G_p(z)} \frac{G_f(z)}{1 - G_f(z)}.
\]

\subsection{Discrete State Space}

\todo{Stuff from skript p. 244.}

\subsection{Deadbeat Controller Design}
\todo{Stuff from skript p. 256.}
