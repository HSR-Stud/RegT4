% !TeX root = ../RegT4.tex
% vim: ts=2 sw=2 spell:

\section{Control in State Space}

\begin{figure}
	\centering
	\resizebox{\linewidth}{!}{
		% vim: ts=2 sw=2 et:
\begin{tikzpicture}[very thick]
  \matrix[
    column sep = 6mm, row sep = 4mm,
  ]{
    &&&& \node[rtbox] (D) {\(\mathbf{D}\)}; \\

    \node[rtsum] (R) {};
      & \node[rtsplit] (U) {};
        \node[below = 0mm of U] {\(\mathbf{u}\)};
      & \node[rtbox] (B) {\(\mathbf{B}\)};
      & \node[rtsum] (dX) {};
        \node[above = 0mm of dX] {\(\dot{\mathbf{x}}\)};
      & \node[rtint] (I) {};
      & \node[rtsplit] (X) {};
        \node[above = 0mm of X] {\(\mathbf{x}\)};
      & \node[rtbox] (C) {\(\mathbf{C}\)};
      & \node[rtsum] (Y) {}; \\[1mm]

    &&&& \node[rtbox] (A) {\(\mathbf{A}\)};
      & \node[rtsplit] (Xb) {}; \\
    &&&& \node[rtbox] (K) {\(\mathbf{K}\)}; \\
  };

  \draw[<-] (R) node[left = 3mm, yshift = -2mm] {\(+\)} -- ++(-1,0) node[left] {\(\mathbf{r}\)};

  \draw[->]
    (R) -- (U)
    (U) edge (B)
    (B) edge (dX)
    (dX) edge (I)
    (I) -- (X)
    (X) edge (C)
    (C) edge (Y)
    (Y) -- ++(1,0) node[right] {\(\mathbf{y}\)}
  ;

  \draw[->] (U) |- (D);
  \draw[->] (D) -| (Y);

  \draw[->]
    (X) edge (Xb)
    (Xb) edge (A)
    (Xb) |- (K);
  \draw[->] (A) -| (dX);
  \draw[->] (K) -| (R) node[below = 3mm, xshift = -2mm] {\(-\)};


\end{tikzpicture}

	}
	\caption{
		Pole placement through the gain matrix \(\mx{K}\).
		\label{fig:ss-pole-placement}
	}
\end{figure}

\subsection{Pole Placement}

\subsubsection{Gain Matrix}

Consider the state space model
\begin{align*}
	\dot{\vec{x}}(t) & = \mx{A} \vec{x}(t) + \mx{B} \vec{u}(t), \\
	\vec{y}(t)       & = \mx{C} \vec{x}(t) + \mx{D} \vec{u}(t),
\end{align*}
and assume that we can somehow always know the state \(\vec{x}\). Then we add a feedback loop from the state to the input with the \emph{gain matrix} \(\mx{K}\) (see figure \ref{fig:ss-pole-placement}). This introduces a new term for the desired value \(\vec{r}\) and changes the state space equation to
\[
	\dot{\vec{x}} = \left(\mx{A} - \mx{B}\mx{K}\right) \vec{x} + \mx{B}\vec{r}.
\]
The addition of \(\mx{K}\) allow to change the eigenvalues of the system matrix and thus move the poles. Therefore, the poles of the new matrix \(\tilde{\mx{A}} := \mx{A} - \mx{B}\mx{K}\) are called the regulator poles. To allow an arbitrary placement of poles the system needs to be completely controllable. Without any additional construction, the gain matrix is basically a proportional controller (German: \textsl{P-Regler}).

% When using the pole placement for stability it is important to keep in mind the following points:
% \begin{itemize}
% 	\item All poles must be on the LHP,
% 	\item All transients should decay fast,
% 	\item The system should be sufficiently damped, for example by setting \(\zeta > 1/\sqrt{2}\),
% 	\item \texttt{TODO}.
% \end{itemize}

\subsubsection{Solve for the Gain Matrix}

Now, if we wish to bring the poles of the closed loop system to the locations \(\mu_1, \mu_2, \ldots, \mu_n\), in the most general case we can use the fact that
\begin{align*}
	\det\left(s\mx{I} - \tilde{\mx{A}}\right)
		&= \det\left(s\mx{I} - \mx{A} + \mx{B}\mx{K}\right) \\
		&= (s - \mu_1)(s - \mu_2) \cdots (s - \mu_n) \\
		&= s^n + \alpha_1 s_{n-1} + \dots + \alpha_{n-1} s + \alpha_n,
\end{align*}
and solve for the entries of \(\mx{K}\) by comparing coefficients.

If the state space representation is in the diagonal form (decoupled, eigenmodes), then finding \(\vec{K}\) is as simple as solving for the entries of \(\mx{K}\):
\[
	\mx{A} - \mx{B} \mx{K} =
	\underbrace{
		\begin{bmatrix}
			\mu_1 \\
			& \ddots \\
			& & \mu_n
		\end{bmatrix}
	}_{\mx{M}}.
\]
Sometimes it is possible to do \(\mx{K} = \mx{B}^{-1} (\mx{A} - \mx{K})\), but more often the system is overdetermined and we are free to choose some entries of \(\mx{K}\) (which is even easier).

If the state space representation is in the controllable canonical form, and 
\[
	\mx{K} = \begin{bmatrix} k_1 & k_2 & \ldots & k_n \end{bmatrix}
\]
since \(\tilde{\mx{A}} = \mx{A} - \mx{B}\mx{K} = \)
{\small
	\[
		\begin{bmatrix}
			0 & 1 & 0 & \dots & 0 \\
			0 & 0 & 1 & \dots & 0 \\
			\vdots & \vdots & \vdots & \ddots & \vdots \\
			0 & 0 & 0 & \dots & 1 \\
			-a_n -k_1 & -a_{n-1} - k_2 & -a_{n-2} - k_3& \dots & -a_1 - k_n
		\end{bmatrix},
	\]
}
which means that
\[
	G(s) = \frac{
		b_{n-1} s^{n-1} + \dots + b_1 s + b_0
	}{
		s^n + (a_{n-1} + k_n) s^{n-1} + \dots + (a_0 + k_1)
	},
\]
the poles are found by comparing the coefficients of the denominator:
\begin{align*}
	s^n + (a_{n-1} + k_n) &s^{n-1} + \dots + (a_0 + k_1) \\
		&= (s - \mu_1)(s - \mu_2) \cdots (s - \mu_n).
\end{align*}

\subsection{Linear Quadratic Regulator (LQR)}

A problem of the pole placement approach is that it is not always obvious which poles should be moved and where. The \emph{linear quadratic regulator} (LQR) also called \emph{quadratic optimal regulator} solves this problem.

In the LQR design we define a performance index to minimize:
\[
	J = \int_0^\infty \left(
		\vec{x}^* \mx{Q} \vec{x} + \vec{u}^* \mx{R} \vec{u}
	\right) \, dt,
\]
where \(\mx{Q}\) and \(\mx{R}\) are positive definite (or semidefinite) Hermitian or real symmetric matrices. The \(\mx{Q}\) matrix determines the importance of the relative error, whereas the \(\mx{R}\) matrix determines the cost (in energy) of modifying the state.  Since in the pole placement \(\vec{u} = -\mx{K}\vec{x}\) the problem becomes
\[
	\argmin_{\mx{K}}
	\int_0^\infty \vec{x}^* \left(
		\mx{Q} + \mx{K}^* \mx{R} \mx{K}
	\right) \vec{x} \, dt.
\]
Then, the expression can be further developed by using the fact that
\[
	\vec{x}(t) = e^{(\mx{A} - \mx{B}\mx{K}) t} \vec{x}(0),
\]
which when substituted into the previous expression gives
\[
	\vec{x}(0)^* \underbrace{\int_0^\infty \left[
			e^{(\mx{A} - \mx{B}\mx{K}) t}
		\right]^* \left(
			\mx{Q} + \mx{K}^* \mx{R} \mx{K}
		\right) e^{(\mx{A} - \mx{B}\mx{K}) t}
		\, dt}_{\mx{S}(\mx{K})}
	\, \vec{x}(0).
\]
The minimization problem is completed by computing
\[
	\text{minimize } J \iff
	0 = \frac{\partial J}{\partial K_{ij}} =
	\vec{x}^*(0) \frac{\partial \mx{S}}{\partial K_{ij}} \vec{x}(0).
\]
It is possible to show that the resulting optimal gain matrix is
\[
	\mx{K} = \mx{R}^{-1} \mx{B}^* \mx{S}.
\]

\subsection{PI-Control}

\subsection{Observer}

\subsection{\texttt{MATLAB} Reference}

\subsubsection{Solve for the Gain Matrix}
