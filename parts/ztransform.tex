% !TeX root = ../RegT4.tex
% vim: ts=2 sw=2 spell:

\section{\(z\)-Transform}

\subsection{Definition}

The \(z\)-transform of a sequence \((f_n)\) and its inverse transform (sometime written as \(\mathcal{Z}, \mathcal{Z}^{-1}\)) are defined to be
\[
	F(z) = \sum_{k = 0}^\infty f_k z^{-k},
	\quad
	f_n = \frac{1}{2\pi i} \ointctrclockwise_C F(z) z^{n-1} \, dz,
\]
where \(z\) is a complex number and \(C\) is a counterclockwise path encircling the origin and entirely in the region of convergence.

\subsection{Relation to the Laplace Transform}

The \(z\)-transform is related to the Laplace transform, when a sequence is generated by sampling a function. In other words with a sampling time \(T\) we can construct a continuous time function
\[
	f(t) = \sum_k f_k \delta(t - kT)
\]
of which is then possible to take the Laplace transform resulting in
\[
	\laplace \left\{ \sum_k f_n \delta(t - kT) \right\}
	= \sum_{k=0}^\infty f_k e^{-kTs}
	= \sum_{k=0}^\infty f_k z^{-k},
\]
where in the last step the substitution \(z = e^{sT}\) was made. Therefore the multiplication by \(z^{-1}\) can interpreted as adding a delay of \(T\).

\begin{table*}
	\renewcommand{\arraystretch}{1.7}
	\begin{tabularx}{\linewidth}{
			l >{\(\displaystyle}l<{\)} >{\(\displaystyle}X<{\)}
			>{\(\displaystyle}l<{\)} >{\(\displaystyle}l<{\)}
		}
		\toprule

		\textbf{Property} & \textbf{Time Domain} & z\textbf{-Domain} &
		\textbf{Sequence} & \textbf{\(z\)-Tr.} \\

		\cmidrule(r){1-3} \cmidrule(l){4-5}

		Linearity & \sum_i a_i f_k(k) & \sum_i a_i F_i (z) &
		\delta(k) & 1 \\

		Forward time shift & f(k + n) & z^n F(z) - z^n f(0) - \dots - z f(n-1) &
		\varepsilon(k) & \frac{z}{z - 1} \\

		Backwards time shift & f(k - n) & z^{-n} F(z) &
		k & \frac{z}{(z - 1)^2} \\

		Convolution & f_1 * f_2 = \sum_{i = 0}^k f_1(k) f_2(k - i) & F_1(z) F_2(z) &
		a^k & \frac{z}{z - a} \\

		\bottomrule
	\end{tabularx}
	\caption{
		Useful rules and transform pairs of the \(z\)-transform. All sequences can be converted into time-domain functions by using a Dirac comb. Some are easier for example \(k \leadsto kT\).
		\label{tab:z-transform}
	}
\end{table*}
