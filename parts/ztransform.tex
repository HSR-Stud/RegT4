% !TeX root = ../RegT4.tex
% vim: ts=2 sw=2 spell:

\section{\(z\)-Transform}

\subsection{Definition}

The \(z\)-transform of a sequence \((f_n)\) is defined to be
\[
	F(z) = \sum_{k = 0}^\infty f_k z^{-k},
\]
where \(z\) is a complex number. The inverse transform is
\[
	f_n = \frac{1}{2\pi i} \ointctrclockwise_C F(z) z^{n-1} \, dz,
\]
where \(C\) is a counterclockwise path encircling the origin and entirely in the region of convergence (ROC).

\subsection{Relation to the Laplace Transform}

The \(z\)-transform is related to the Laplace transform, when a sequence is generated by sampling a function. In other words with a sampling time \(T\) we can construct a continuous time function
\[
	f(t) = \sum_k f_k \delta(t - kT)
\]
of which is then possible to take the Laplace transform resulting in
\[
	\laplace \left\{ \sum_k f_n \delta(t - kT) \right\}
	= \sum_{k=0}^\infty f_k e^{-kTs}
	= \sum_{k=0}^\infty f_k z^{-k},
\]
where in the last step the substitution \(z = e^{sT}\) was made. Therefore the multiplication by \(z^{-1}\) can interpreted as adding a delay of \(T\).
